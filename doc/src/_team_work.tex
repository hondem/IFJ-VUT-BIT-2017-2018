\section{Práce v týmu}
Tým se skládal ze čtyř členů, kteří byli odhodlaní začít
pracovat již před zadáním projektu. Projekt byl verzován pomocí systému Git hostovaným na serveru GitHub. jednotkové testy poté automaticky spouštěny na službě Travis CI. Pro sdílenou komunikaci byla využívána služba Gitter IM, pro správu úkolů poté GitHub Issues.

Schůzky byly svolávány každý týden a pokaždé bylo jasně rozděleno, co který člen týmu bude mít v následujícím týdnu za úkol. Veškeré problémy byly řešeny ihned při vzniku aby bylo zabráněno vzniku nesrovnalostí mezi implementovanými moduly. 

V druhé polovině projektu započala práce na integračních testech, těch bylo veřejně vytvořeno přes 600 za přispění i ostatních týmů. V rámci těchto testů bylo také vytvořeno vývojové prostředí s debuggerem pro snažší kontroly generování cílového kódu.

\subsection{Rozdělení práce}
\begin{itemize}
    \item Martin Kobelka - lexikální a syntaktická analýza, definice testů
    \item Tomáš Pazdiora - syntaktická analýza výrazů, dokumentace
    \item Josef Kolář - generátor kódu, sémantické kontroly, optimalizace výrazů
    \item Son Hai Nguyen - architektura, interprocedurální optimalizace kódu
\end{itemize}