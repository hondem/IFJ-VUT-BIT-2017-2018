\section{Práce v týmu}
Náš tým se skládal ze čtyř členů, kteří byli odhodlaní začít
pracovat už od prázdnin.

Jako verzovací systém jsme použili GIT
na serveru Github z důvodů nulové ceny pro studenty a jednoduché
implementace. Pro vzájemnou komunikaci jsme zřidili kanál na komunikačním serveru
gitter do kterého jsme se přihlašovali pomocí svých github účtů. Pro rozebírání
důležitějších témat jsme použili github issues, pro neformálnější komunikaci
společnou konverzaci na serveru facebook.

Schůzky jsme organizovali každý týden a na každé si stanovili objem práce, který by měl
být hotový do konce týdne.

Z počátku jsme všechny části vymýšleli společně a vše si kontrolovali,
aby později nemolo dojít k nedorozuměním. Když už byly vymyšleny
principy na kterých budou dané moduly fungovat, došlo k rozdělení
odpovědností a každý dostal k dokončení jednu či více částí.

\subsection{Rozdělení práce}
\begin{itemize}
    \item Martin Kobelka - Lexikální analyzátor, syntaktický analyzátor, definice testů
    \item Tomáš Pazdiora - Syntaktická analýza výrazů, dokumentace
    \item Josef Kolář - Generátor kódů, implementace tabulky symbolů
    \item Son Hai Nguyen - Optimalizace kódu, poloautomatické generování pravidel
\end{itemize}