\section{Práce v~týmu}
Tým se skládal ze čtyř členů, kteří byli odhodlaní začít
pracovat již před zadáním projektu. Projekt byl verzován pomocí systému \textbf{Git} hostovaným na serveru \textbf{GitHub}. Jednotkové testy poté automaticky spouštěny na službě \textbf{Travis CI}. Pro sdílenou komunikaci byla využívána služba \textbf{Gitter IM}, pro správu úkolů poté \textbf{GitHub Issues}.

Schůzky byly svolávány \textbf{každý týden} a pokaždé bylo jasně rozděleno, co který člen týmu bude mít v~následujícím týdnu za úkol. Veškeré problémy byly \textbf{řešeny ihned při vzniku}, aby bylo zabráněno vzniku nesrovnalostí mezi implementovanými moduly.

V~druhé polovině projektu započala práce na \textbf{integračních testech}, těch bylo veřejně vytvořeno přes \textbf{600 jednotek} za přispění i několika ostatních týmů. V~rámci těchto testů bylo také vytvořeno \textbf{vývojové prostředí s~debuggerem} pro snažší kontrolu generování ekvivaletně fungujícího cílového kódu.

\subsection{Rozdělení práce}
\begin{itemize}
    \item Martin Kobelka - lexikální a syntaktická analýza, definice testů
    \item Josef Kolář - generátor kódu, sémantické kontroly, optimalizace výrazů
    \item Son Hai Nguyen - architektura, interprocedurální optimalizace kódu
    \item Tomáš Pazdiora - syntaktická analýza výrazů, dokumentace
\end{itemize}