\section{Struktura projektu}
Jazyk IFJ17 je podmnožinou jazyka Free Basic. Jazyk není case-sensitiove,
což znamená že nerozlišuje malá a velká písmena. Je jazykem staticky typovaným.
\subsection{Lexikální analýza}
Lexikální analyzátor je implementací deterministického konečného
automatu. Jeho úlohou je rozpoznávat jazykové lexémy a reprezentovat
je jako tokeny.

Lexikální analyzátor je rozdělen do tří modulů \texttt{lexer.c}, \texttt{lexer\_fsm.c} a \texttt{token.c}
. Modul \texttt{token.c} obsahuje implementaci abstraktního datového typu token. Modul \texttt{lexer\_fsm.c}
obsahuje implementaci deterministického konečného automatu. Modul \texttt{lexer.c} poskytuje funkci
\texttt{lexer\_next\_token}, která řídí činnost deterministického konečného automatu a
která vrací další token.

\subsection{Synataktická analýza}
Naším úkolem bylo implementovat syntaxí řízený překlad. To znamená
že syntaktická analýza řídí čínnost překladače.

Pro syntaktickou analýzu programu jako celku byla použita syntaktická
analýza shora dolů, konkrétně metoda rekurzivního sestupu. Implementace se nachází v modulu \texttt{parser.c}

Pro analýzu výrazů není analýza shora dolů příliš vhodná, proto byla
použita metoda zdola nahorů v našem případě založená na precedenční
syntaktické analýza implementovaná v souboru \texttt{parser\_expr.c}.

Přepínání mezi metodami je realizováno následujícím mechanismem. Na celý program je použita funkce \texttt{parser_parse}
z modulu \texttt{parser.c}. V případě že je jako neterminál očekáván výraz, je volána funkce \texttt{parser_parse_expr}
z modulu \texttt{parser\_expr}.c, která provede precedenční syntaktickou analýzu výrazu.

\subsection{Precedenční analýza výrazů}
\todo{Describe a case-by-case analysis}
\subsection{Interpret}
\todo{Describe something about interpret}
